The representative household solves the following problem:
\begin{align*}
  V(a;X) &= \max_{c,a'} \frac{c^{1-\sigma}}{1-\sigma} + \beta \mathbb{E}V(a';X')
  \\
  \text{s.t. }& 
  c+a' - (1-\delta)a = Aa^{\alpha}
  \\
  &
  a' - (1-\delta)a \geq \phi I_{ss}
\end{align*}
where $V$ is the value function of a household. The value function's arguments are wealth $a$ and the aggregate state $X$. $c$ is consumption and $\sigma$ is the risk-aversion parameter.
$I_{ss}$ is the steady-state investment level. $\phi$ is the parameter that governs the degree of the irreversibility. $\delta$ is the depreciation rate, and $\alpha$ is the capital share in the production function $F = Aa^{\alpha}$.
An apostrophe indicates a future allocation.
The aggregate state $X$ is as follows
\begin{align*}
  X = [K,A].
\end{align*}
$K$ is the aggregate capital stock, satisfying $a=K$ in equilibrium due to the capital market clearing. $A$ is TFP that follows the log AR(1) process:
\begin{align*}
  log(A') = \rho log(A) + \sigma\epsilon,\quad\sigma\sim N(0,1).
\end{align*}
\subsection*{Optimality conditions}
The Largrangian is as follows:
\begin{align*}
  \mathcal{L} = \frac{c^{1-\sigma}}{1-\sigma} + \beta \mathbb{E}V(a';S') + \mu(Aa^{\alpha} + (1-\delta)a- c-a' ) + \lambda(a' - (1-\delta)a - \phi I_{ss}),
\end{align*}
where $\lambda$ is the Lagrange multiplier.
The first-order optimality conditions are as follows:
\begin{align*}
  [c]:&\quad c^{-\sigma} = \mu
  \\
  [a']:&\quad \mu = \beta \mathbb{E}V_{1}(a';S') + \lambda
  \\
  [a]:&\quad V_{1}(a;S) = \mu(\alpha A a^{\alpha-1} + (1-\delta)) - (1-\delta)\lambda
\end{align*}
Then, we obtain
\begin{align*}
  c^{-\sigma} -\lambda= \beta \mathbb{E}\left[(c')^{-\sigma}(\alpha A' (a')^{\alpha-1}+(1-\delta)) -(1-\delta)\lambda'\right].
\end{align*}