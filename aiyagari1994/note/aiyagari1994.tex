\subsection*{Recursive formulation}
A continuum of measure one of households is considered, and they live forever in discrete time.
Each household consumes, and saves. The labor endowment $z$ is exogenously determined by an $AR(1)$ process. The recursive formulation of the household's problem is as follows:
\begin{align*}
  v(a,z;\Phi) &= \max_{c,a'} \quad log(c) + \beta \mathbb{E}_{z}v(a',z';\Phi)
  \\
  \text{s.t.}&\quad c + a' = w(\Phi)z + a(1+r(\Phi))
  \\
  &\quad a'\geq \underline{a} = 0
  \\
  &\quad log(z') = \rho log(z) + \sigma\sqrt{1-\rho^{2}}\epsilon,\quad \epsilon\sim N(0,1)
  % &\quad z' \sim \Gamma(z'|z)\quad(\text{Markov chain})
\end{align*}
where apostrophe indicates future allocations. $\Phi$ is the joint distribution of the individual states $(a,z)$. The borrowing limit $\underline{a}$ is given as $0$.
The prices $w(\Phi)$ and $r(\Phi)$ are determined at the competitive labor and capital input markets.
Now we consider a production sector that operates using the CRS Cobb-Douglas production function:
\begin{align*}
	% f(k_{t},n_{t}) = 
	\max_{K,L} A K^{\alpha}L^{1-\alpha} - (r(\Phi)+\delta)K - w(\Phi)L
\end{align*}
The aggregate TFP $A=1$ fixed.
The competitive input markets are cleared at the prices $(w,r)$:
\begin{align*}
	\text{\underline{Supply}}&\quad\text{ \underline{Demand}}
	\\
	\text{[Capital market]}\quad\qquad \int ad\Phi &= K
	\\
	\text{[Labor market]}\quad\qquad \int z d\Phi &= L
\end{align*}
The parameters levels are set as in Aiyagari (1994), as follows:
\begin{align*}
  \rho = 0.9,\quad \sigma = 0.2,\quad \alpha = 0.36,\quad \beta = 0.96, \quad \delta = 0.08.
\end{align*}
The idiosyncratic labor endowment process is discretized by Tauchen method using 7 grid points covering $\pm3$ standard-deviation range.