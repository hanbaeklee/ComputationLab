\documentclass[12pt]{article}
\usepackage{amsfonts}
\usepackage{amssymb}
\makeatletter
\usepackage{amsmath}
\usepackage{amsthm}
\usepackage{setspace}
\usepackage{geometry}
\usepackage{bbm}
\usepackage{graphicx}
\usepackage{color}
\usepackage{lscape}
\usepackage{float}
\usepackage{pdfpages}
\usepackage{subcaption}
\usepackage{mathpazo}
\usepackage{booktabs,caption}
\usepackage[flushleft]{threeparttable}
\usepackage{natbib}
\usepackage[breaklinks]{hyperref}
\hypersetup{
    colorlinks = true,
    linkcolor=blue,
    citecolor=blue,
    urlcolor=cyan
    }
\newcommand\numberthis{\addtocounter{equation}{1}\tag{\theequation}}
\newtheorem{prop}{Proposition}
\newtheorem{lemma}{Lemma}
\newtheorem{theorem}{Theorem}
\newtheorem{corollary}{Corollary}
\newtheorem{definition}{Definition}
\newtheorem{assumption}{Assumption}
\newtheorem{conjecture}{Conjecture}
\newcommand{\Mod}[1]{\ (\mathrm{mod}\ #1)}
\renewcommand\qedsymbol{$\blacksquare$}
\setcounter{MaxMatrixCols}{10}
\geometry{left=6em, right=6em, top=6em,bottom=6em}
\pagestyle{plain}
\usepackage[toc,page]{appendix}
\newcommand{\nocontentsline}[3]{}
\newcommand{\tocless}[2]{\bgroup\let\addcontentsline=\nocontentsline#1{#2}\egroup}

\begin{document}
\setstretch{1.25}\small\normalsize%
\doublespace

\clearpage
\section*{A canonical RBC model}
\subsection*{Recursive formulation}
The representative household solves the following problem:
\begin{align*}
  V(a;S) &= \max_{c,a',L} log C + 
  \theta log(1-L)+ \beta \mathbb{E}V(a';S')
  \\
  \text{s.t. }& 
  (1+\tau^{c})c+a' = (1+(1-\tau^{r})r(S))a + (1-\tau^{w})w(S)L
  % \\
  % &
  % a' - (1-\delta)a \geq \phi I_{ss}
\end{align*}
where the aggregate state $S$ is as follows
\begin{align*}
  S = [K,A].
\end{align*}
$K$ is the aggregate capital stock. $A$ is TFP that follows the log AR(1) process:
\begin{align*}
  log(A') = \rho log(A) + \sigma\epsilon,\quad\sigma\sim N(0,1).
\end{align*}
$c$ is consumption, and $a$ is the wealth in the beginning of a period. 

% \subsection*{Optimality conditions}
% The Largrangian is as follows:
% \begin{align*}
%   \mathcal{L} = \frac{c^{1-\sigma}}{1-\sigma} + \beta \mathbb{E}V(a';S') + \mu(Aa^{\alpha} + (1-\delta)a- c-a' ) + \lambda(a' - (1-\delta)a - \phi I_{ss}),
% \end{align*}
% where $\lambda$ is the Lagrange multiplier.
% The first-order optimality conditions are as follows:
% \begin{align*}
%   [c]:&\quad c^{-\sigma} = \mu
%   \\
%   [a']:&\quad \mu = \beta \mathbb{E}V_{1}(a';S') + \lambda
%   \\
%   [a]:&\quad V_{1}(a;S) = \mu(\alpha A a^{\alpha-1} + (1-\delta)) - (1-\delta)\lambda
% \end{align*}
% Then, we obtain
% \begin{align*}
%   c^{-\sigma} -\lambda= \beta \mathbb{E}\left[(c')^{-\sigma}(\alpha A' (a')^{\alpha-1}+(1-\delta)) -(1-\delta)\lambda'\right].
% \end{align*}
\end{document}
