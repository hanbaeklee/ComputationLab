\subsection*{Recursive formulation}
The representative household solves the following problem:
\begin{align*}
  V(a;S) &= \max_{c,a'} \frac{c^{1-\sigma}}{1-\sigma} - 
  \frac{\eta}{1+\frac{1}{\chi}}L^{1+\frac{1}{\chi}}+ \beta \mathbb{E}V(a';S')
  \\
  \text{s.t. }& 
  (1+\tau^{c})c+a' = (1+(1-\tau^{r})r(S))a + (1-\tau^{w})w(S)L
  % \\
  % &
  % a' - (1-\delta)a \geq \phi I_{ss}
\end{align*}
where the aggregate state $S$ is as follows
\begin{align*}
  S = [K,A].
\end{align*}
$K$ is the aggregate capital stock. $A$ is TFP that follows the log AR(1) process:
\begin{align*}
  log(A') = \rho log(A) + \sigma\epsilon,\quad\sigma\sim N(0,1).
\end{align*}
$c$ is consumption, $a$ is the wealth in the beginning of a period. 
% $\phi$ is the parameter that governs the degree of the partial irreversibility.

% \subsection*{Optimality conditions}
% The Largrangian is as follows:
% \begin{align*}
%   \mathcal{L} = \frac{c^{1-\sigma}}{1-\sigma} + \beta \mathbb{E}V(a';S') + \mu(Aa^{\alpha} + (1-\delta)a- c-a' ) + \lambda(a' - (1-\delta)a - \phi I_{ss}),
% \end{align*}
% where $\lambda$ is the Lagrange multiplier.
% The first-order optimality conditions are as follows:
% \begin{align*}
%   [c]:&\quad c^{-\sigma} = \mu
%   \\
%   [a']:&\quad \mu = \beta \mathbb{E}V_{1}(a';S') + \lambda
%   \\
%   [a]:&\quad V_{1}(a;S) = \mu(\alpha A a^{\alpha-1} + (1-\delta)) - (1-\delta)\lambda
% \end{align*}
% Then, we obtain
% \begin{align*}
%   c^{-\sigma} -\lambda= \beta \mathbb{E}\left[(c')^{-\sigma}(\alpha A' (a')^{\alpha-1}+(1-\delta)) -(1-\delta)\lambda'\right].
% \end{align*}